% Document packages / layout
\documentclass[a4paper,11pt]{article}

% Load packages 
\usepackage{amsfonts}
\usepackage{amsopn}
\usepackage{amsmath}
\usepackage{amsthm}
\usepackage{amssymb}
\usepackage{mathrsfs}
\usepackage{mathtools}
\usepackage{multirow}
\usepackage{tabularx}
\usepackage{hhline}
\usepackage{scalerel,stackengine}
\usepackage{arydshln}
\usepackage[normalem]{ulem}
\usepackage{soul}
\usepackage{tikz}
\usepackage{pgf}
\usepackage{collcell}
\usepackage[utf8]{inputenc}
\usepackage{geometry}
\usepackage{physics}

\usepackage{graphbox}
\usepackage{setspace}

\usepackage{fullpage}

\usepackage{hyperref}
\usepackage{cleveref}

% \allowdisplaybreaks

% Quality of Life Packages
% \usepackage{enumerate} %styles of enumeration
% \usepackage{cprotect} %verbatim in caption
% \usepackage{fancyvrb}
% \usepackage{subfig}
% \usepackage{float}
% \usepackage{todonotes}

% Bibliography Packages
\usepackage[maxbibnames=99, backend=biber]{biblatex}
\usepackage{breakcites}
\usepackage[shortlabels]{enumitem}


% Enable manipulating numbers in tables
\newcolumntype{E}{>{\collectcell\usermacro}c<{\endcollectcell}}
\newcommand\usermacro[1]{\pgfmathparse{10000*#1}\pgfmathprintnumber\pgfmathresult}

%% Algorithm style, could alternatively use algpseudocode
\usepackage{algorithm}
\usepackage{comment}
\usepackage{algpseudocode}

% Add a serial/Oxford comma by default.
\newcommand{\creflastconjunction}{, and~}




% Load Notation

% TODO package and associated commands. Remove before submission.
\usepackage[prependcaption]{todonotes}
\newcommand{\abhi}[1]{\todo[linecolor=green,backgroundcolor=green!25,bordercolor=green]{#1}}
\makeatletter 
\@mparswitchfalse%
\makeatother
\normalmarginpar

\newcommand{\bs}{\boldsymbol}
\def\ds{\displaystyle}
\newcommand{\mb}[1]{\mathbb{#1}}
\newcommand{\mc}[1]{\mathcal{#1}}
\newcommand{\ms}[1]{\mathscr{#1}}

\newcommand{\mbf}[1]{\ensuremath{\mathbf{#1}}}
\newcommand{\ut}[1]{\ensuremath{\tilde{#1}}}
\newcommand{\ui}[1]{\ensuremath{\hat{#1}}}
\newcommand{\bui}[1]{\ensuremath{\hat{\boldsymbol{#1}}}}
\newcommand{\bu}[1]{\ensuremath{\boldsymbol{#1}}}
\newcommand{\mbu}[1]{\ensuremath{\mathbf{#1}}}

\newcommand{\mat}[1]{\mathbf{#1}}
\renewcommand{\vec}[1]{\mathbf{#1}}

\newcommand{\R}{\mathbb{R}}

\newcommand{\e}{\varepsilon}

\DeclareMathOperator*{\argmax}{arg\,max}
\DeclareMathOperator*{\argmin}{arg\,min}

% Variables
\newcommand{\state}{u}
\newcommand{\test}{p}
\newcommand{\invparam}{m}
\newcommand{\obs}{y}
\newcommand{\noise}{\eta}
\newcommand{\robustparam}{\theta}
\newcommand{\avgrobustparam}{\overline{\robustparam}}
\newcommand{\optrobustparam}{\robustparam^{\rm opt}}

\newcommand{\vzero}{\mbf{0}}

\newcommand{\design}{\xi}
\newcommand{\optdesign}{\design^{\rm opt}}
\newcommand{\alldesign}{\design^{\rm all}}
\newcommand{\policy}{\mbf{p}}
\newcommand{\scalarpolicy}{p}
\newcommand{\policyweights}{\mbf{w}}
\newcommand{\scalarpolicyweights}{w}
\newcommand{\optpolicy}{\mbf{p}^{\rm opt}}
\newcommand{\robustoptpolicy}{\mbf{p}_{\robustparam}^{\rm opt}}

\newcommand{\stochoedgrad}{\mbf{g}}
\newcommand{\scalarstochoedgrad}{g}

\newcommand{\baseline}{b}
\newcommand{\optbaseline}{b^{\rm opt}}

%% Adjoint based differentiation
\newcommand{\hatstate}{\hat{\state}}
\newcommand{\hattest}{\hat{\test}}
\newcommand{\hatinvparam}{\hat{\invparam}}

\newcommand{\checkstate}{\check{\state}}
\newcommand{\checktest}{\check{\test}}
\newcommand{\checkinvparam}{\check{\invparam}}

\newcommand{\tildestate}{\tilde{\state}}
\newcommand{\tildetest}{\tilde{\test}}
\newcommand{\tildeinvparam}{\tilde{\invparam}}

% Operators
\newcommand{\Sim}{\mc{S}}
\newcommand{\Obs}{\mc{Q}}
\newcommand{\POM}{\mc{F}}
\newcommand{\JPOM}{\mc{J}}

\newcommand{\KLD}{D_{\rm KL}}
\newcommand{\EKLD}{\overline{\KLD}}

\newcommand{\G}{\mc{G}}

\newcommand{\Hm}{\mc{H}_{\rm m}}
\newcommand{\ppHm}{\widetilde{\mc{H}}_{\rm m}}
\newcommand{\I}{\mc{I}}

\newcommand{\utility}{\mc{U}}
\newcommand{\stochobj}{\mc{Y}}
\newcommand{\proj}{P}

\newcommand{\lowrankig}{\mc{U}^{\rm LRIG}}
\newcommand{\lowrankeig}{\mc{U}^{\rm LREIG}}

\newcommand{\Expectation}{\mathbb{E}} % expectation
\renewcommand{\Probability}{\mathbb{P}} % probability
\newcommand{\budget}{\mbf{Z}}

\newcommand{\Bilaptheta}{\mc{K}}
\newcommand{\bilaptheta}{\kappa}

% Sizes
\newcommand{\Ndata}{{\rm N_{d}}}
\newcommand{\Nrobust}{{\rm N_{\robustparam}}}
\newcommand{\Nsaa}{{\rm N_{SAA}}}
\newcommand{\Nens}{{\rm N_{ens}}}
\newcommand{\Nbudget}{{\rm N_{b}}}
\newcommand{\Nsamples}{{\rm N_{s}}}

% Spaces
\newcommand{\statespace}{\ms{U}}
\newcommand{\testspace}{\ms{U}'}
\newcommand{\paramspace}{\ms{M}}
%\newcommand{\dataspace}{\ms{Y}}
\newcommand{\dataspace}{\mathbb{R}^{\Ndata}}
\newcommand{\designspace}{\{0, 1\}^{\Ndata}}
\newcommand{\domain}{\Omega}
\newcommand{\budgetdesignspace}{Z_{\Nbudget}}
\newcommand{\policyspace}{[0, 1]^{\Ndata}}
\newcommand{\robustspace}{\Theta}
\newcommand{\finiterobustspace}{\overline{\robustspace}}

% Functions
\newcommand{\inp}[2]{\left\langle #1, #2 \right\rangle}
%\newcommand{\vinp}[2]{\inp{#1}{#2}_{\testspace}}

% Probability Distributions
%% prior
\newcommand{\prior}{\pi_{\rm pr}}
\newcommand{\priormeasure}{\mu_{\rm pr}}
\newcommand{\priormean}{\invparam_{\rm pr}}
\newcommand{\priorcov}{\mc{C}_{\rm pr}}
\newcommand{\priorcovinv}{\priorcov^{-1}}
\newcommand{\priorcovsqrt}{\priorcov^{1/2}}
\newcommand{\priorcovinvsqrt}{\priorcov^{-1/2}}
\newcommand{\cameronmartin}{\mc{E}}

%% noise
\newcommand{\likelihood}{\pi_{\rm like}}
\newcommand{\noisecov}{\Gamma_{\rm n}}
\newcommand{\noisecovinv}{\noisecov^{-1}}

%% posterior
\newcommand{\post}{\pi_{\rm post}}
\newcommand{\postmeasure}{\mu_{\rm post}}
\newcommand{\lapostmeasure}{\mu_{\rm post}^{\rm LA}}
\newcommand{\postmean}{\invparam_{\rm post}}
\newcommand{\postcov}{\mc{C}_{\rm post}}
\newcommand{\postcovinv}{\mc{C}_{\rm post}^{-1}}


% Add Bibliography Sources 
%\addbibresource[glob]{Bib/*.bib}
\addbibresource{references.bib}

\begin{document}

\title{%
  Continuous Sensor Placement for the Optimal Design of Bayesian Inverse Problems
}
\author{%
  Abhijit Chowdhary
}
\maketitle

% Add table of Contents
%\tableofcontents

\section{Infinite Dimensional Bayesian Inverse Problems}
\label{sec:infinite-dimensional-bayesian-inverse-problems}

Let us first define the inverse problem to set some common notation.

We consider the problem of inferring a distribution on a parameter of interest from
measurement data and a given simulation model. In particular, we intend to learn about
an inversion parameter $\invparam \in \paramspace$ from data $\obs \in \dataspace$ from
a model $\POM(\invparam)$ that related $\invparam$ to the data via the relationship
\begin{equation} \label{eq:data-model}
  \obs = \POM(\invparam) + \noise
\end{equation}
Here, $\noise$ is a random vector quantifying measurement and model error independent of
$\invparam$, distributed as $\mc{N}(0, \noisecov)$, where $\noisecov$ is a
positive-definite covariance operator. The operator $\POM: \paramspace \to \dataspace$
is what we denote as the parameter-to-observable map and typically involves the
composition of a simulation model and observation operator. That is
\begin{equation}
  \POM(\invparam) = \Obs \circ \Sim(\invparam)
\end{equation}
$\Sim: \paramspace \to \statespace$ is a simulation model that maps the parameter to a
state space $\statespace$ and $\Obs: \statespace \to \dataspace$ is an observation
operator that maps the state space to the data space. We will assume $\paramspace$ and
$\statespace$ are infinite-dimensional Hilbert spaces that represent function spaces.
Finally, let us define $\domain = {\rm domain}(\state)$ to be the common domain of all
functions $\state \in \statespace$. 

We denote by $\likelihood(\obs | \invparam)$ the distribution of data $\obs$ given 
$\invparam$ under the model \cref{eq:data-model}. Considering the additive regime
detailed, we have that $\obs | \invparam \sim \mc{N}(\POM(\invparam), \noisecov)$, that is:
\begin{equation} \label{eq:likelihood}
  \likelihood(\obs | \invparam) 
  \propto
  \exp \left(
    -\frac{1}{2} \norm{\obs - \POM(\invparam)}_{\noisecovinv}^2
  \right).
\end{equation}
Finally, we assume a Gaussian prior measure $\priormeasure = \mc{N}(\priormean,
\priorcov)$ with $\priormean \in \paramspace$ and $\priorcov$ a strictly positive
self-adjoint operator of trace class. We define the Cameron-Martin space of
$\priormeasure$ as the subspace $\cameronmartin = {\rm range}(\priorcovsqrt)$.
This is endowed with the inner product
\begin{equation}
  \label{eq:cameron-martin-inner-product}
  \ip{x}{y}_{\cameronmartin} = \ip{\priorcovinvsqrt x}{\priorcovinvsqrt y}_{\paramspace}
  , \quad x, y \in \cameronmartin
\end{equation}
Finally, we assume $\priormean \in \cameronmartin$.

These assumptions on the data model and prior, along with Bayes' rule, define a
posterior measure $\postmeasure(\invparam | \obs)$ on $\paramspace$ given by the
Radom-Nikodym derivative
\begin{equation}\label{eq:bayes}
  \dv{\postmeasure}{\priormeasure}
  \propto \likelihood(\obs | \invparam)
\end{equation}

\section{OED for a Binary Design}
\label{sec:binary-oed}
Let us now turn to the question of the optimal sensor placement for the above Bayesian
inverse problem. The binary formulation is typically developed as follows. Suppose we
have an initial set of $\Ndata$ sensors distributed throughout $\Omega$ that can be
activated or deactivated. Then, let $\design \in \designspace$ be a binary vector where
an active $i$th sensor corresponds to $\design_i = 1$ and an inactive sensor corresponds
to $\design_i = 0$. Finally, let us assume that we have a sensor budget of $\Nbudget$.
Then, the optimal design $\optdesign$ is the solution of the optimization problem
\begin{gather}
  \label{eq:binary-oed}
  \max_{\design \in S(N_b)} \utility(\design)
  \\
  S(N_b) = \{\design \in \designspace : |\design| = \Nbudget\}
\end{gather}
where $\utility: \designspace \to \R$ is some utility function that quantifies the
quality of the inverse problem solved according to the given design. 

\section{OED for a Continuous Design}
\label{sec:continuous-oed}
Now, let us formulate a continuous design problem for the optimal sensor placement. We
would like to place $\Nbudget$ sensors somewhere in $\Omega$ to maximize some utility.
Instead of considering which $\Nbudget$ sensors out of a pool of $\Ndata$ sensors to
activate, let us instead consider a continuous design space where $\design_i \in
\Omega$. That is, $\design_i$ corresponds to the coordinates of the location of the
$i$th sensor in $\Omega$. For example, if $\Omega \in [0,1]$ and $\Nbudget = 1$, then
the design $\design = (0.5)$ could correspond to placing a sensor at the center of the
domain.

With this formulation, we would instead seek an optimal design $\optdesign \in
\Omega^{\Nbudget}$ that solves the optimization problem
\begin{equation}
  \label{eq:continuous-oed}
  \max_{\design \in \Omega^{\Nbudget}} \utility(\design)
\end{equation}
There are a couple of critical benefits to this formulation. 
\begin{itemize}
  \item The continuous design space allows for a more flexible design space that can
    capture more complex sensor placements, away from the 'grid' of the binary design.
  \item The continuous search space enables gradient-based optimization methods to be
    used to solve the optimization problem. These are often much more efficient than
    combinatorial optimization methods used for \cref{eq:binary-oed}.
  \item Optimization occurs in space which has dimension $\Nbudget \times \dim(\Omega)$,
    which is typically much smaller than the $\Ndata$-dimensional space of the binary
    design. Especially as $\dim(\Omega)$ is typically $2$ or $3$.
\end{itemize}
However, it does come with a few drawbacks.
\begin{itemize}
  \item Gradient based optimization would require $\utility_{\design}$, which may not be
    simple to formulate. In the binary setting, $\design$ enters $\Obs$ simply through a
    selection matrix; here it's a coordinate parameter of an interpolation operator.
  \item The observation error model $\mathcal{N}(0, \noisecov)$ would have to be defined
    in the budget space, and the errors would have to selected in light of the fact that
    the sensor could be placed anywhere in $\Omega$. Additionally, the notion of
    correlation between sensors would have to be carefully considered.
\end{itemize}

%%%%%%%%%%%%%
%\appendix
%%%%%%%%%%%%%
% Load Appendix sections 
%\input{appendix}

%%%%%%%%%%%%%
\printbibliography
%%%%%%%%%%%%%

\end{document}
